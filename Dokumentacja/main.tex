\documentclass{article}
\usepackage[polish]{babel}
\usepackage[utf8]{inputenc}
\usepackage{geometry}
\usepackage{graphicx}
\usepackage[T1]{fontenc}
\usepackage{float}
\geometry{a4paper, margin=1in}

\title{
    \vspace{-2cm}
    \begin{flushright}
        \includegraphics[width=6cm]{photos/logoWSIiZ.PNG} 
    \end{flushright}
    \vspace{1cm}
    \textbf{Wyższa Szkoła Informatyki i Zarządzania w Rzeszowie} 
    \vspace{1cm}
}
\author{
    \Large Kolegium Informatyki Stosowanej \\
    \normalsize Kierunek: Informatyka \\
    \vspace{1cm}
    \normalsize Filip Walat \\
    \normalsize Nr albumu studenta: w67204 \\
    \vspace{1cm}
    \LARGE System Zarządzania Parkingiem \\
    \vspace{0.5cm}
    \Large Projekt Programowanie Obiektowe
}
\date{
    \vspace{10cm}
    \begin{center}
        \Large Rzeszów 2023
    \end{center}
    \vspace{7cm}
}

\begin{document}

\maketitle

\tableofcontents

\clearpage

\section{Opis Założeń Projektu}
Projekt ma na celu stworzenie kompleksowego systemu obsługi parkingu opartego na platformie C\#. Celem tego przedsięwzięcia jest zaprojektowanie i implementacja efektywnego systemu, który umożliwi zarządzanie ruchem pojazdów na parkingu oraz zapewni wygodę użytkownikom. Projekt ten stanowi integralną część studiów informatycznych, skupiając się na zastosowaniu praktycznych umiejętności programistycznych w środowisku C\#.

\section{Wymagania Funkcjonalne}

Użytkownik ma możliwość sprawdzenia dostępności miejsc parkingowych poprzez wybranie opcji "Check Availability". System wykorzystuje metodę \texttt{CheckAvailability} z klasy \texttt{Parking} do przedstawienia liczby dostępnych miejsc parkingowych.

System umożliwia wprowadzenie pojazdu na parking. Użytkownik podaje numer rejestracyjny, kolor pojazdu oraz wybiera typ pojazdu (samochód, motocykl, autobus). Na podstawie tych danych tworzony jest odpowiedni obiekt pojazdu (\texttt{Car}, \texttt{Motorcycle}, \texttt{Bus}), a następnie wykorzystywana jest metoda \texttt{EnterParking} z klasy \texttt{Parking} do zarejestrowania pojazdu na parkingu.

Użytkownicy mogą również zarejestrować wyjazd pojazdu z parkingu, podając numer rejestracyjny pojazdu. Funkcjonalność ta jest realizowana przez metodę \texttt{ExitParking} klasy \texttt{Parking}, która aktualizuje stan miejsc parkingowych.

System oferuje opcję wyświetlenia ogólnych informacji o projekcie, co jest realizowane poprzez prostą instrukcję wyświetlającą na ekranie.

Funkcja \texttt{exit} umożliwia użytkownikowi bezpieczne zakończenie pracy z aplikacją. Wybór tej opcji kończy działanie programu, zapisując wszelkie zmiany w stanie parkingu lub w danych pojazdów.

\section{Wymagania Niefunkcjonalne}

Projekt wykorzystuje zorientowaną obiektowo hierarchię klas (\texttt{Vehicle}, \texttt{Car}, \texttt{Motorcycle}, \texttt{Bus}), co zapewnia łatwą rozszerzalność i modułowość systemu. Dziedziczenie umożliwia dodawanie nowych typów pojazdów bez konieczności modyfikacji istniejącego kodu, co zwiększa skalowalność systemu i ułatwia jego rozwój.

System zaprojektowano z myślą o minimalizacji czasu odpowiedzi, kluczowej dla płynnego działania w środowisku czasu rzeczywistego. Metody takie jak \texttt{CheckAvailability}, \texttt{EnterParking}, i \texttt{ExitParking} są optymalizowane pod kątem szybkiego przetwarzania danych, co przekłada się na wydajność aplikacji.

Klasa \texttt{Parking} efektywnie zarządza stanem parkingowym, przechowując informacje o zajętych i dostępnych miejscach. Wykorzystanie zaawansowanych struktur danych umożliwia efektywną manipulację stanem i szybki dostęp do danych.

Struktura projektu została przygotowana z myślą o łatwej integracji mechanizmów zabezpieczających, takich jak autentykacja użytkownika czy szyfrowanie danych, co podnosi poziom zabezpieczeń i niezawodności systemu.

Kod źródłowy jest zgodny z konwencjami nazewnictwa i zasadami SOLID, co zapewnia jego wysoką czytelność i ułatwia zarządzanie projektem oraz wprowadzanie ewentualnych modyfikacji.

Wykorzystanie dziedziczenia i polimorfizmu pozwala na efektywne reużywanie kodu i elastyczne traktowanie różnych typów pojazdów. Klasy pochodne od \texttt{Vehicle} implementują unikalne metody, pozwalając na różnorodne traktowanie pojazdów, przy jednoczesnym zachowaniu spójnego interfejsu.

Projekt został zorganizowany w sposób umożliwiający łatwe przeprowadzanie testów jednostkowych i integracyjnych, co jest kluczowe dla zapewnienia stabilności i niezawodności systemu.
\clearpage
\section{Opis Struktury Projektu}

Projekt Systemu Zarządzania Parkingiem skupia się na zorientowanej obiektowo architekturze, co ułatwia zarządzanie różnymi typami pojazdów i interakcje z parkingiem. Centralnym elementem jest klasa \texttt{Parking}, która zarządza miejscami parkingowymi, a także wprowadzaniem i wyjazdem pojazdów. Klasy pojazdów, takie jak \texttt{Car}, \texttt{Motorcycle} i \texttt{Bus}, dziedziczą z abstrakcyjnej klasy \texttt{Vehicle}, co pozwala na polimorficzne traktowanie różnych typów pojazdów. Poniżej przedstawiono diagram klas, który ilustruje relacje między głównymi komponentami systemu.

\begin{figure}[H]
\centering
\includegraphics[width=0.75\textwidth]{photos/diagram.png}
\caption{Diagram klas systemu zarządzania parkingiem.}
\end{figure}

Struktura projektu jest zaprojektowana w taki sposób, aby maksymalizować ponowne wykorzystanie kodu i ułatwić rozszerzanie systemu o nowe funkcjonalności.

\section{Opis Techniczny Projektu}

Projekt został zrealizowany w języku C\#, co zapewnia szerokie możliwości w zakresie programowania obiektowego i zarządzania danymi. Do zarządzania projektem i kodem źródłowym wykorzystano środowisko Visual Studio Code z dodatkowymi wtyczkami, takimi jak PlantUML dla generowania diagramów klas UML, oraz Git jako system kontroli wersji.

System jest zaprojektowany z myślą o niskich wymaganiach sprzętowych, co czyni go dostępnym na większości współczesnych komputerów i serwerów. Minimalne wymagania to:
\begin{itemize}
    \item Procesor: 1 GHz lub szybszy.
    \item Pamięć RAM: 512 MB dla klienta, 2 GB dla serwera.
    \item Przestrzeń na dysku: 100 MB.
    \item System operacyjny: Windows 7 lub nowszy, Linux, MacOS.
\end{itemize}

Projekt wykorzystuje mechanizm zarządzania danymi oparty na bazie danych w pliku tekstowym (txt), co pozwala na prostą i efektywną manipulację danymi bez potrzeby korzystania z zewnętrznych systemów DBMS. Taki wybór umożliwia łatwą portowalność i minimalizuje wymagania sprzętowe oraz konfiguracyjne. Struktura plików tekstowych jest zaprojektowana w taki sposób, aby umożliwić szybkie odczytywanie i zapisywanie stanu miejsc parkingowych oraz informacji o pojazdach, co zapewnia wysoką wydajność działania systemu.
\clearpage
\section{Harmonogram Realizacji Projektu}

Harmonogram realizacji projektu został zaplanowany z wykorzystaniem diagramu Gantta, który ilustruje kluczowe etapy rozwoju projektu, ich zależności czasowe oraz alokację zasobów. Poniżej przedstawiono diagram Gantta dla projektu System Zarządzania Parkingiem.

\begin{figure}[H]
\centering
\includegraphics[width=\textwidth]{photos/gant1.png}
\caption{Diagram Gantta projektu System Zarządzania Parkingiem.}
\end{figure}
\begin{figure}[H]
\centering
\includegraphics[width=\textwidth]{photos/gant2.png}
\caption{Diagram Gantta projektu System Zarządzania Parkingiem.}
\end{figure}
\begin{figure}[H]
\centering
\includegraphics[width=0.45\textwidth]{photos/gant3.png}
\caption{Diagram Gantta projektu System Zarządzania Parkingiem.}
\end{figure}

\section{Repozytorium i System Kontroli Wersji}

Projekt wykorzystuje system kontroli wersji Git, co umożliwia skuteczne zarządzanie historią zmian kodu źródłowego. Repozytorium kodu znajduje się na platformie GitHub pod adresem:\\ \url{https://github.com/filwalu/ProjektOOP} i będzie dostępne publicznie do dnia 30.09.2024. Bezpieczne\\ połączenie z repozytorium zabezpieczono za pomocą pary kluczy SSH. Poniżej przedstawiono opis\\ użytych poleceń Git:

\begin{enumerate}
    \item \texttt{git init} - inicjalizacja nowego repozytorium Git.
    \item \texttt{git clone [URL]} - klonowanie repozytorium przy użyciu SSH.
    \item \texttt{git add -Av} - dodawanie zmian do kolejki commitów.
    \item \texttt{git status} - sprawdzanie statusu zmian.
    \item \texttt{git commit -m "[wiadomość]"} - commitowanie zmian z opisem.
    \item \texttt{git push} - wysyłanie zmian do zdalnego repozytorium przez SSH.
\end{enumerate}

\section{Prezentacja Warstwy Użytkowej Projektu}

Projekt Systemu Zarządzania Parkingiem oferuje intuicyjny i prosty w obsłudze interfejs użytkownika, który umożliwia szybkie zarządzanie miejscami parkingowymi oraz monitorowanie dostępności przestrzeni parkingowej. Interfejs użytkownika został zaprojektowany z myślą o zapewnieniu maksymalnej użyteczności i dostępności funkcji.

Aplikacja umożliwia użytkownikom wykonanie następujących akcji:
\begin{itemize}
    \item Sprawdzenie dostępności miejsc parkingowych.
    \item Rejestracja wjazdu i wyjazdu pojazdów.
    \item Zarządzanie danymi pojazdów.
\end{itemize}

Interfejs skupia się na minimalizmie i łatwości nawigacji, co pozwala na szybkie odnalezienie potrzebnych informacji i funkcji.

W tej sekcji zostaną umieszczone zrzuty ekranu przedstawiające kluczowe funkcjonalności aplikacji oraz jej interfejs użytkownika.

\begin{figure}[H]
\centering
\includegraphics[width=\textwidth]{photos/mainmenu.png}
\caption{Widok głównego menu aplikacji.}
\end{figure}

W tej sekcji omówione zostaną kluczowe funkcjonalności aplikacji Systemu Zarządzania Parkingiem oraz instrukcje dotyczące ich używania.

\begin{itemize}
    \item \textbf{Sprawdzanie dostępności miejsc parkingowych} (\textit{Check Availability}):
    Użytkownik może sprawdzić dostępność miejsc parkingowych dla samochodów, wybierając opcję "1". 

    \begin{figure}[H]
    \centering
    \includegraphics[width=\textwidth]{photos/avail.png}
    \caption{Opcja 1 - Check Availability.}
    \end{figure}

    \item \textbf{Wprowadzanie pojazdów na parking}:
    Użytkownik może wprowadzić pojazd na parking, wybierając opcję "2". Następnie należy podać numer rejestracyjny pojazdu, jego kolor oraz wybrać typ pojazdu (1 - Samochód, 2 - Motocykl, 3 - Autobus). Na podstawie podanych informacji, system tworzy odpowiedni obiekt pojazdu i rejestruje go w systemie parkingowym. 
        \begin{figure}[H]
        \centering
        \includegraphics[width=\textwidth]{photos/enter.png}
        \caption{Opcja 2 - Enter Parking.}
        \end{figure}
    \clearpage
    \item \textbf{Wyjazd pojazdów z parkingu}:
    Użytkownik może wyjechać pojazdem z parkingu, wybierając opcję "3". Następnie należy podać numer rejestracyjny pojazdu. Na podstawie podanej informacji, system usuwa pojazd z rejestru parkingowego. 
    \begin{figure}[H]
        \centering
        \includegraphics[width=\textwidth]{photos/exit.png}
        \caption{Opcja 3 - Exit Parking.}
        \end{figure}
    \item \textbf{Dodatkowe informacje}:
    Użytkownik wybierając opcję "4"\ wyświetla dodatkowe informacje o systemie.
        \begin{figure}[H]
        \centering
        \includegraphics[width=\textwidth]{photos/info.png}
        \caption{Opcja 4 - General Information.}
        \end{figure} 
\item \textbf{Koniec}:
    Użytkownik wybierając opcję "5"\ zamyka aplikacje.
\end{itemize}

Sekcja ta zawiera informacje na temat interakcji z użytkownikiem, w tym otrzymywanych komunikatów, alertów oraz sposobu obsługi błędów. Poniżej przedstawiono wybrane komunikaty wyświetlane przez aplikację:

\begin{itemize}
    \item \texttt{"Available spots: [True lub False]"} - informacja mówiąca czy znajdują się obecnie wolne miejsca parkingowe.
    \item \texttt{"Enter registration number:"} - monit o wprowadzenie numeru rejestracyjnego pojazdu.
    \item \texttt{"Enter color:"} - prośba o podanie koloru pojazdu.
    \item \texttt{"Choose vehicle type [1] Car [2] Motorcycle [3] Bus:"} - wybór typu pojazdu do wprowadzenia na parking.
    \item \texttt{"Invalid vehicle type selected."} - komunikat o błędnym wyborze typu pojazdu.
    \item \texttt{"Vehicle entered the parking."} - informacja o pomyślnym wprowadzeniu pojazdu na parking.
\end{itemize}

\section{Podsumowanie}

W ramach projektu Systemu Zarządzania Parkingiem wykonano szereg prac, w tym:
\begin{itemize}
    \item Projektowanie i implementacja klas reprezentujących pojazdy (samochody, motocykle, autobusy) oraz parking.
    \item Rozwój funkcjonalności zarządzania miejscami parkingowymi, w tym sprawdzanie dostępności, rejestracja wjazdu i wyjazdu pojazdów.
    \item Przygotowanie interfejsu użytkownika w konsoli do interakcji z systemem.
    \item Wykorzystanie plików tekstowych do prostego zarządzania danymi.
\end{itemize}

W dalszej kolejności planowane są następujące prace rozwojowe:
\begin{itemize}
    \item Zastosowanie klasycznej bazy danych SQL, np. PostgreSQL, dla lepszego zarządzania danymi i wydajności.
    \item Konteneryzacja aplikacji z wykorzystaniem Docker, co ułatwi wdrożenie i skalowalność.
    \item Implementacja systemu płatności.
    \item Usprawnienia interfejsu użytkownika, w tym rozwój graficznego interfejsu użytkownika (GUI) dla lepszej dostępności i ergonomii.
    \item Rozbudowa obsługi wyjątków dla zwiększenia stabilności i niezawodności aplikacji.
    \item Implementacja możliwości generowania raportów w formacie CSV z danych przechowywanych w systemie dla ułatwienia analizy i zarządzania.
\end{itemize}

\clearpage 
\listoffigures 

\clearpage 
\begin{thebibliography}{9}

\bibitem{gitpro}
Scott Chacon, Ben Straub. \textit{Pro Git}. Apress, 2014. Dostępne online: \url{https://git-scm.com/book/en/v2}

\bibitem{dotnetdocs}
Dokumentacja Microsoft .NET. Microsoft, dostęp online: \url{https://docs.microsoft.com/en-us/dotnet/}

\bibitem{csharpstudy}
Albahari, Joseph, Albahari, Ben. \textit{C\# 7.0 in a Nutshell: The Definitive Reference}. O'Reilly Media, 2017.

\end{thebibliography}

\end{document}

